%======================================================================
\chapter{Introduction}
%======================================================================
This report entertains the idea of generating random graphs. More specifically, the graphs will be used as training data for a Generative Adversarial Network tasked with Text-to-Image-to-Text Translation \cite{hu2021text}. However, rather than using generic text as output, the GAN will be tasked with converting the images to a tactile writing system, more specifically braille. And while this would prove extremely beneficial for the visually impaired, before any real-world applications can be entertained, the first barrier of entry must be solved. In this case, the first obstacle was obtaining an appropriate dataset.

\hfill

As having a large and refined data set is one of most important aspects to any data science problem, generating a dataset of random graphs is a fundamental function for the entire text-to-image-to-text translation procedure and should be documented accordingly. The following report will provide an in-depth look into various facets of the graph generation and regeneration pipeline. To start off, the report will cover an overview of the motivation and problem domain to further set the scene as to what challenges are being solved and why. The following chapter will provide be an overview of the methodology, which will discuss topics such as the generation algorithms and their use cases. In addition, examples of theorized solutions and discarded approaches will also be brushed upon. Subsequently, there will be a deep dive into the implementation details such as how the entire chart generation and regeneration pipelines are structured. Moreover, a detailed explanation of what is needed for future application support such as additional libraries and graph types will also be examined. The final two chapters discuss the important design decisions made along the way and the rationale behind them, as well as the state of project going forward and any unfinished tasks respectively.

\hfill

That being said, the main objective of this project is to create a system for automatically generating 2-D plots in different styles using a multitude of unique Python libraries. This system will then be used to create training data for the network pipeline, as reliably gathering tens of thousands of randomized graphs is a tedious and difficult process. The main deliverable for the project is a Python project executed through a Jupyter notebook for generating an endless amount of images from the visualization libraries Altair, Bokeh, and plotnine respectively. These images representing the 2-D plots, along with the raw values and parameters used to generate each chart, will then be saved to a specified directory for future testing and model assessment. 