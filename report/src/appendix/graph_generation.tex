% An appendix
%======================================================================
\chapter[Graph Generation]{Graph Generation}
\label{AppendixA}
% Tip 4: Example of how to get a shorter chapter title for the Table of Contents 
%======================================================================

\lstset{
basicstyle=\small\ttfamily,
columns=flexible,
breaklines=true
}

\section{Data Generation}
\begin{lstlisting}
generated_graphs = deepcopy(graphs)
graph_type_id, current_graph_type = 0, None
for (graph_type, library), num_occurences in occurences.items():
    if (current_graph_type != graph_type):
        graph_type_id = 0
        current_graph_type = graph_type
        
    graphs_list = generated_graphs[graph_type].setdefault('graphs', [])
    for _ in range(num_occurences):
        data = generate_data(graph_type)
        graphs_list.append({
            'id': graph_type_id,
            'library': library,
            'data': data,
        })
        graph_type_id += 1
\end{lstlisting}

\section{Data Stylization}
\begin{lstlisting}
for (graph_type, graph_object) in generated_graphs.items():
    for graph_content in graph_object['graphs']:
        library, num_repeats = graph_content['library'], graph_content['data'].get('num_repeats', 1)
        graph_content['styles'] = generate_styles(graph_type, library, num_repeats)
        graph_content['data'].pop('num_repeats', None)
\end{lstlisting}

\section{Graph Creation \& Graph Exportation}
\begin{lstlisting}
for (graph_type, graph_object) in generated_graphs.items():
    for graph_content in graph_object['graphs']:
        library = graph_content['library']
        graph = create_graph(graph_type, library, graph_content)
        
        id = graph_content['id']
        file_name = '{graph_type}_{library}_{id}'.format(graph_type=graph_type, library=library, id=id)
        
        export_graph_data(graph_content['data'], 'output/{path}/{file_name}.{file_type}'
            .format(file_name=file_name, path='data', file_type='json'))
        export_graph_styles(graph_content['styles'], 'output/{path}/{file_name}.{file_type}'
            .format(file_name=file_name, path='styles', file_type='json'))
        export_graph_image(graph, library, 'output/{path}/{file_name}.{file_type}'
            .format(file_name=file_name, path='images', file_type='png'))
\end{lstlisting}

\section{Graph Regeneration}
\begin{lstlisting}
for graph_filepath in valid_input_graphs:
  graph_filename = graph_filepath.split('.', maxsplit=1)[0]
  graph_type, library, id = graph_filename.split('_', maxsplit=3)
  graph_content = {}
  
  for folder in input_folders:
    with open('{dir}/{folder}/{file_name}'.format(dir=INGESTION_DIR, folder=folder, file_name=graph_filepath)) as f:
      graph_content[folder] = convert_from_serializable(json.load(f))
  
  graph = create_graph(graph_type, library, graph_content)
  export_graph_image(graph, library, '{dir}/{path}/{file_name}.{file_type}'
    .format(dir=INGESTION_DIR, file_name=graph_filename, path='images', file_type='png'))
\end{lstlisting}