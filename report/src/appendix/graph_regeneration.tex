% An appendix
%======================================================================
\chapter[Graph Regeneration]{Graph Regeneration}
\label{AppendixB}
% Tip 4: Example of how to get a shorter chapter title for the Table of Contents 
%======================================================================

\lstset{
basicstyle=\small\ttfamily,
columns=flexible,
breaklines=true
}

\section{Graph Regeneration}
\begin{lstlisting}
for graph_filepath in valid_input_graphs:
  graph_filename = graph_filepath.split('.', maxsplit=1)[0]
  graph_type, library, id = graph_filename.split('_', maxsplit=3)
  graph_content = {}
  
  for folder in input_folders:
    with open('{dir}/{folder}/{file_name}'.format(dir=INGESTION_DIR, folder=folder, file_name=graph_filepath)) as f:
      graph_content[folder] = convert_from_serializable(json.load(f))
  
  graph = create_graph(graph_type, library, graph_content)
  export_graph_image(graph, library, '{dir}/{path}/{file_name}.{file_type}'
    .format(dir=INGESTION_DIR, file_name=graph_filename, path='images', file_type='png'))
\end{lstlisting}