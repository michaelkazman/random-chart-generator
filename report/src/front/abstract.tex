The work being developed is based around the idea of creating a Generative Adversarial Network (GAN) for translating computerized images of 2-D graphs to tactile graphics \cite{goodfellow2014generative}. Tactile graphics are the standard format used by the Braille Authority of North America for conveying non-textual information to people who are blind or visually impaired \cite{tactile_graphics}. While this may include tactile representations of pictures, maps, graphs, diagrams, and other images, however, in the context of this project it will only represent 2-D graphs. Graphs of varying types will need to be generated, including but not limited to area graphs, bar charts, box plots, bubble plots, contour plots, error bar plots, histograms, KDE plots, line graphs, scatter plots, and violin plots. As having a reliable and robust set of training data is the fundamental step in any data science problem, this project is concerned with generating an endless amount of fully randomized training data in the form of 2-D graphs. These graphs will be plotted using random data generated through a multitude of ways and stylized with various libraries and themes. Once generated, the graphs will be fed into a GAN for translation in a future project alongside a graduate student of the same supervisor.